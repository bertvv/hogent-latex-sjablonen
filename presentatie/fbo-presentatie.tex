%%----------------------------------------------------------------------------
%% Presentatie HoGent Bedrijf en Organisatie
%%----------------------------------------------------------------------------
%% Auteur: Bert Van Vreckem [bert.vanvreckem@hogent.be]

\documentclass{beamer}

%==============================================================================
% Aanloop
%==============================================================================

%---------- Packages ----------------------------------------------------------

\usepackage{graphicx,multicol}
\usepackage{comment,enumerate,hyperref}
\usepackage{amsmath,amsfonts,amssymb}
\usepackage{tikz}
\usepackage[dutch]{babel}
\usepackage[utf8]{inputenc}
\usepackage{multirow}
\usepackage{eurosym}
\usepackage{listings}
\usepackage[T1]{fontenc}
\usepackage{lmodern}
\usepackage{textcomp}

%---------- Configuratie ------------------------------------------------------

\usetikzlibrary{arrows,shapes,backgrounds,positioning,shadows}

\usetheme{hogent}

%---------- Commando-definities -----------------------------------------------



%---------- Info over de presentatie ------------------------------------------

\title[Korte titel]{Een lange presentatietitel over meerdere lijnen}
\author{Bert {Van Vreckem} \small(\href{mailto:bert.vanvreckem@hogent.be}{bert.vanvreckem@hogent.be})}
\date{\today}

%==============================================================================
% Inhoud presentatie
%==============================================================================

\begin{document}

%---------- Front matter ------------------------------------------------------

% Dia met het HoGent logo
\HoGentLogo

% Titeldia met faculteitslogo
\begin{frame}[plain]
  \titlepage
\end{frame}

%---------- Inhoud ------------------------------------------------------------

% Dia voor sectiekop, voorbeeld met een afbeelding onderaan de pagina
\part{Deel 1}
\partframe{%
  Het eerste deel
  \vfill
  \includegraphics[width=3cm]{logo/HG-woordmerk-inv}
}

\section{Sectie 1}

\subsection{Subsectie 1.1}

\begin{frame}
  \frametitle{Titel}

  \begin{itemize}
  \item Lijn 1
  \item Lijn 2
  \item Lijn 3
  \end{itemize}
\end{frame}

\begin{frame}
  \frametitle{Twee kolommen}

  \begin{columns}[c]

  \column{.5\textwidth}
    \begin{itemize}
    \item Lijn 1
    \item Lijn 2
    \item Lijn 3
    \end{itemize}

  \column{.5\textwidth}
    \begin{itemize}
    \item Lijn 1
    \item Lijn 2
    \item Lijn 3
    \end{itemize}

  \end{columns}
\end{frame}

\subsection{Subsectie 1.2}

\part{Deel 2}
\partframe{Het tweede deel}

\section{Sectie 1}

\begin{frame}
  \frametitle{Tekstblok}

  Voorbeeld van een \alert{blok}

  \begin{block}{Titel}
    Inhoud van het \alert{blok}
  \end{block}

\end{frame}


\begin{frame}
  \frametitle{Tekstblok 2}

  Een blok met nadruk:

  \begin{alertblock}{Titel}
    Inhoud van het blok
  \end{alertblock}

\end{frame}

%---------- Back matter -------------------------------------------------------

\end{document}
