%%----------------------------------------------------------------------------
%% Examenopgave HoGent Bedrijf en Organisatie
%%----------------------------------------------------------------------------
%% Auteur: Bert Van Vreckem [bert.vanvreckem@hogent.be]

\documentclass{exam}

\usepackage{graphicx}
\usepackage[dutch]{babel}
\usepackage[utf8]{inputenc}
\usepackage{hyperref}
\usepackage{rotating}
\usepackage{multirow}

% Lettertype en symbolen
\usepackage[T1]{fontenc}
\usepackage{lmodern}
\usepackage{textcomp}
\usepackage{wasysym,amssymb,latexsym,amsfonts}

\setlength{\parindent}{0pt}

\newif\ifsolution % Vlag die aanduidt of de oplossing afgedrukt moet worden

%%----------------------------------------------------------------------------
%% Info over het examen
%%----------------------------------------------------------------------------

\newcommand{\academiejaar}{2014-2015}   % vb. 2012-2013
\newcommand{\examenperiode}{1$^{\mbox{e}}$}         % vb. 1e, 2e
\newcommand{\examenmaand}{januari 2015} % vb. januari 2013
\newcommand{\examendatum}{8 jan 2015}
\newcommand{\examenuur}{10:30}

\newcommand{\opleiding}{Toegepaste informatica}
\newcommand{\olod}{Systeembeheer--Linux} % vb. Algoritmen
\newcommand{\reeks}{Reeks 1}            % vb. Reeks 1, Reeks 2, Inhaalexamen

\newcommand{\campus}{Schoonmeersen}      % vb. Schoonmeerssen, Aalst
\newcommand{\klassen}{3TI}               % vb. 3TI, ITM, ...
\newcommand{\lectoren}{Bert Van Vreckem}

%% Voorbeeldoplossing? (ja -> \solutiontrue; nee -> \solutionfalse)
\solutionfalse

%%----------------------------------------------------------------------------
%% Lay-out, hoofding
%%----------------------------------------------------------------------------

%% In principe zijn hier geen wijzigingen nodig!

\renewcommand{\familydefault}{\sfdefault} % Schreefloos lettertype

\pointsinmargin
\marginpointname{pt}
\addpoints

\pagestyle{head}
\runningheadrule
\extraheadheight{1cm}
\firstpageheader{\includegraphics[height=2cm]{FBO-NL}}%
{}%
{\small{Valentin Vaerwyckweg 1\\9000 Gent\\T +32 9 243 22 00\\fbo@hogent.be\\www.hogent.be/fbo}}
\runningheader{\olod, \reeks \ifsolution : Voorbeeldoplossing\fi}%
{}%
{p. \thepage/\numpages}

\begin{document}

\ifsolution
\begin{center}
\LARGE{\textbf{VOORBEELDOPLOSSING}}
\end{center}
\fi

\begin{tabular}{|l|l|l|}
\hline
\multicolumn{3}{|p{.95\textwidth}|}{\textbf{Academiejaar \academiejaar{} -- \examenperiode{} examenperiode (\examenmaand) \hfill \reeks}}\tabularnewline
\hline
\multicolumn{2}{|p{.65\textwidth}|}{Opleiding en afstudeerrichting: \opleiding} & Examendatum: \examendatum \tabularnewline
\multicolumn{2}{|p{.65\textwidth}|}{Opleidingsonderdeel: \olod} & Aanvangsuur: \examenuur \tabularnewline
\multicolumn{2}{|p{.65\textwidth}|}{Campus: \campus{}} & Klas(sen): \klassen \tabularnewline
\multicolumn{2}{|p{.65\textwidth}|}{Lector(en): \lectoren} &  \tabularnewline
\hline
\multicolumn{3}{|l|}{Naam en voornaam student:}\tabularnewline
\multicolumn{3}{|l|}{}\tabularnewline
\hline
Geboortedatum student: & \multicolumn{2}{|p{.4\textwidth}|}{Studentennummer: }\tabularnewline
 & \multicolumn{2}{|p{.4\textwidth}|}{}\tabularnewline
\hline
Lector bij wie de student de onderwijsactiviteit volgde: \enspace \hrulefill & \multicolumn{2}{|p{.4\textwidth}|}{Lesgroep waarin de student de onderwijsactiviteiten volgde:: }\tabularnewline
 & \multicolumn{2}{|p{.4\textwidth}|}{}\tabularnewline
\hline
\multicolumn{3}{|l|}{Behaald resultaat: \_\_\_\_\_ op \numpoints{}}\tabularnewline
\hline
\end{tabular}

\vspace{.3cm}

\XBox Tijdens het examen mogen GEEN hulpmiddelen gebruikt worden

\Square Tijdens het examen mogen volgende hulpmiddelen gebruikt worden:
\begin{itemize}
\item
\end{itemize}

\hrulefill

\begin{questions}

\framedsolutions
\ifsolution
  \printanswers
\else
  \noprintanswers
\fi

%%----------------------------------------------------------------------------
%% Examenvragen
%%----------------------------------------------------------------------------
  
\question[1] Volledig en correct invullen hoofding

\question[9] Waarom zijn de bananen krom?
  
\begin{solutionordottedlines}[3cm]
Daarom.
\end{solutionordottedlines}

\end{questions}

\end{document}
